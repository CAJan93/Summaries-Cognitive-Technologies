\documentclass[runningheads]{llncs}
\usepackage{aspPackage}

\newcommand{\papertitle}{Answer Set Programming for Procedural Content Generation}
\newcommand{\authorquote}{Smith \etal}

\begin{document}

\title{A Summary of \papertitle}

\author{Jan Mensch}

\authorrunning{J. Mensch}

\institute{University of Potsdam\\ 
\email{jan.mensch@uni-potsdam.de}\\}


%
\maketitle              % typeset the header of the contribution
% About \textit{\papertitle}:
% Provides both technical, tutorial solution and survey contribution
% technical: map PCG problems to ASP
%tutorial: introduction to ASP and code-level walkthrough of existing PCG systems.  
% survey: review of existing ASP applications with generative purpose


\begin{abstract}
This is a summary of the paper \textit{\papertitle}\footnote{Title abbreviated. The original title is \textit{Answer set programming for procedural con-tent generation: A design space approach.}} \cite{smith2011answer}. In it, \authorquote{} describe how Procedural Content Generation (PCG) can be implemented with ASP\footnote{We assume that the reader is familiar with Answer Set Programming. If that is not the case, we refer the reader to \cite{erdem2016applications}.}. The authors also provide an introduction to ASP, a code-level walkthrough of existing PCG systems and a review of existing ASP applications with generative purpose. Since a full summary of the paper by \authorquote{} would go beyond the scope of this article, this text is focused on a summary of technical aspects. We will do so by giving an introduction to Procedural Content Generation (Section \ref{sec:introduction}) and an overview of the PCG process and why it makes sense to implement generators with ASP (Section \ref{sec:ta}). 

\end{abstract}


\section{Introduction to Procedural Content Generation} \label{sec:introduction}

    As the name suggests, Procedural Content Generation is an automated process of generating content. This generated content is referred to as an artifact. In the context of video games, such an artifact could, for example, be a maze described by logical facts (like the start and finish location and the traversability of the maze). The generation of artifacts is based on a so-called \textit{design space} (also known as \textit{generative space}), which is the space the generator searches to produce an output. Or in other words: The design space of a generator is the set of artifacts the generator could eventually produce, given some (optional) input\footnote{Note that out of the set of all possible artifacts only some will represent an interesting solution.}. This design space is often implicit. 
    
    PCG involves two problems: The production of artifacts and the production of a generative procedure. The former is found by a machine and judged by a human (e.g. by answering the question ``How much fun is it to play this artifact?''). The later is found by human designers and judged by its expressiveness, flexibility, and performance. The problem of coming up with a good PCG-process is that the developer has to propose both, a generator that produces good artifacts and a meta-process that enables human designers to work with this model. We will explain the approach of \authorquote{} to both of these problems in the next section.

\section{Technical Aspects} \label{sec:ta}
% what, how, why?

\subsection{Approach}

    A possible approach towards PCG is to focus on the generation of good artifacts. This  will usually result in many small changes in the generative procedure itself, thus making bigger changes to the procedure (e.g. changing the level of abstraction) difficult. To the contrary, the approach proposed by \authorquote{} is to focus only on describing the design space with a declarative language and to make use of existing solvers, that accept these programs to produce artifacts. The approach can be described as follows: Reason about the design space, implement the design space using the declarative language and pass this program to a solver which outputs artifacts. Afterwards, the designers review the produced artifacts and make changes to the design space, based on their review. Since the design space is implicit \authorquote{} do not expect the designers to find a perfect representation for it in their first attempt. The generative procedure is therefore meant to be iterative and should be repeated until the  designers are satisfied with the artifacts that the generator produces. 
    
    \subsection{Implementation with ASP}
    
    To implement the process described above, \authorquote{} use ASP. The authors justify this decision with the fact that ASP provides a rich, declarative language (\textit{ANSProlog}) and that the paradigm targets NP-complete problems. Since ANSProlog is a declarative language, the designers can easily modify the design space, which is necessary for the iterative process described above. Another advantage of ASP is that it comes with generic solvers\footnote{These solvers are similar to SAT solvers \cite{lin2004assat}. For a pseudocode example of algorithms used in ASP solvers, we would like to refer the reader to \cite[p.~221]{russel}.} and that a designer will not need detailed knowledge about the solvers in order to use them. He/she can focus on describing the design space with ANSProlog, a language that is accepted by most solvers \cite{baral2003knowledge}.  
    
    Another advantage of ANSProlog is that it is able to express both data and logic, ``default values''\footnote{A programmer could express logic like ``assume that it is always dry, except you are encountering rain'' with ANSProlog. The ``default value'' for the moistness would be ``dry''.} and recursion. A code example from the domain of strategy games could be the following snippet: 
    \begin{verbatim}
            hostile(A,B) :- enemy(A,B).
            hostile(A,C) :- enemy(A,B), friend(B,C).
            hostile(A,C) :- friend(A,B), enemy(B,C).
    \end{verbatim}
    The meaning of this code can be interpreted as ``I am hostile to someone if they are my enemy, if they are the friend of one of my enemies, or if they are the enemy of one of my friends''. The snippet also demonstrates the ability of ANSProlog to express recursive behavior. An example could be the following statement: ``B is an enemy of mine. \verb|C| is a friend of \verb|B|, so I am an enemy of \verb|C|. If \verb|D| is a friend of \verb|C|, I am also an enemy of \verb|D|.'' The relationship between ``myself'' and \verb|D| was determined indirectly via recursion. 
    
    The authors also mention ``Shortcomings'' of ANSProlog: The language is not able to describe infinite data structures, which might be considered to be a useful feature, when one wants to generate a potentially infinite world. However, \authorquote{} also point out that this limitation can usually be avoided by breaking up bigger tasks into smaller ones, which only require  finite design spaces (and therefore also only finite data structures).
    

\section{Conclusion} \label{sec:conclusion}

Procedural Content Generation touches on two problems: The creation of good artifacts and the implementation of a fast and flexible generative procedure. In their paper \textit{\papertitle} \authorquote{} provide a framework which can be used as a generative procedure for creating content generators. The strength of the framework is rooted in the power of ASP, which provides a rich language and generic, efficient solvers. 

%\begin{itemize}
%    \item PCG about two things: good artifacts and good process (fast and flexible).
 %   \item Existing applications show that ASP is suitable for the implementations of PCG, especially because of its rich declarative language and efficient solvers.
  %  \item authors provide framework for creating content generators. 
%\end{itemize}

% The development of the application touches on many questions of ... like ...
% It is the hope of \authorquote{} that ...


\bibliographystyle{unsrt}
\bibliography{refs}

\end{document}
