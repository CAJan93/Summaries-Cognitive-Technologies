\documentclass[runningheads]{llncs}
\newcommand{\papertitle}{Modeling and Language Extensions} 
\usepackage{aspPackage}

\usepackage{graphicx}
% Used for displaying a sample figure. If possible, figure files should
% be included in EPS format.
%
% If you use the hyperref package, please uncomment the following line
% to display URLs in blue roman font according to Springer's eBook style:
% \renewcommand\UrlFont{\color{blue}\rmfamily}
\usepackage{mathtools} % for arrows with text over/under it compare https://tex.stackexchange.com/questions/103988/rightarrow-with-text-above-it




\begin{document}
%
\title{A Summary of \papertitle}
%
%\titlerunning{Abbreviated paper title}
% If the paper title is too long for the running head, you can set
% an abbreviated paper title here
%
\author{Jan Mensch}
%
\authorrunning{J. Mensch}
% First names are abbreviated in the running head.
% If there are more than two authors, 'et al.' is used.
%

\institute{University of Potsdam, Am Neuen Palais 10, 14469 Potsdam, Germany\\ 
\email{jan.mensch@uni-potsdam.de}\\
\url{https://www.uni-potsdam.de/}}


%
\maketitle              % typeset the header of the contribution
%
\begin{abstract}
This is a summary of the paper \papertitle. 



%\keywords{First keyword  \and Second keyword \and Another keyword.}
\end{abstract}
%
%
%

\section{Notes}


\begin{itemize}
    \item One-shot solving: Take program and instances, grounding and than solve/serching it. 
    \item multishot solving. Interleave (verschachtelt) grounding and searching. We solve the problem step by step, e.g. by implmenting a step function that simulates the passing of one unit of time.
    \item traveling salesperson problem: A salesperson wants to travel all cities of some set in minimal time without visiting cities twice (spanning (directed) tree with minimal cost). 
    \item differentiation between optimization criteria and constraints that just have to be met.
    \item We can also use :~ $travel(X,Y), link(X,Y,C),~[C,X]$ for optimization. This is called a weak constraint. We are now trying to minimize C and X. \item Emphasize again that things that are obvious to humans must be stated, like that every city only can have on predecessor. 
    \item summary is good
    \item Breaking up program in 4 parts: GENERATE (potential solutions), DEFINE (starting constraints) TEST (against constrainsts) and OPTIMIZE (optimization criteria). 
    \item for one multishot solving is done by defining both the data and the goal situation. 
    \item How to multishot: base program: Called once on init.  check program: Run on every time step. Step: Execute a step.
\end{itemize}

\section{Questions}

\begin{itemize}
    \item What does the predicate place do?
    \item How is TSP solved under the hood? Is it depth first or breadth first?c
    \item Am I allowed to visit a city twice in the given example? No right, since every city can only have one predecessor.
    \item difference between DEFINE and TEST? Both are constraints that my program has to handle. Answer Define are integrity constraints, things that are implicit to humans.
\end{itemize}



\bibliographystyle{unsrt}
\bibliography{refs}

\end{document}

